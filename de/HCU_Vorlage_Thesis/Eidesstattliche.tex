\section*{Eidesstattliche Erklärung}
\addcontentsline{toc}{section}{Eidesstattliche Erklärung}
\markboth{\uppercase{Eidesstattliche Erklärung}}{}

\begin{table}[H]
	\begin{tabular}{rl}
		Name:            & \Nachname \\
		Vorname:         & \Vorname \\
		Matrikelnummer:  & \Matrikel \\
		Studienprogramm: & \Studiengang
	\end{tabular}
\end{table}

Ich versichere, dass ich die vorliegende Thesis mit dem Titel\\
\\
{\Large \Titel} \\
\\
selbstständig und ohne unzulässige fremde Hilfe erbracht habe.\\
\\
Ich habe keine anderen als die angegebenen Quellen und Hilfsmittel benutzt sowie wörtliche und sinngemäße Zitate kenntlich gemacht. Die Arbeit hat in gleicher oder ähnlicher Form noch keiner Prüfungsbehörde vorgelegen.\\
\\
\\
\begin{minipage}[H]{0.49\textwidth}
	Hamburg, der \Abgabe
\end{minipage}
\begin{minipage}[H]{0.49\textwidth}
	\includegraphics[width=0.45\textwidth,trim={0 25mm 0 0}]{Daten/Unterschrift.png}
\end{minipage}

\hrule
\vspace{3mm}

\begin{minipage}[H]{0.49\textwidth}
	Ort, Datum
\end{minipage}
\begin{minipage}[H]{0.49\textwidth}
	Unterschrift
\end{minipage}

\vspace{\fill}

\begin{minipage}[H]{0.49\textwidth}
	\textbf{Vom Prüfungsamt auszufüllen}\\
	{\small Die o.g. Thesis wurde abgegeben am}
	\vspace{3cm}
\end{minipage}
\begin{minipage}[H]{0.49\textwidth}
	\addtolength{\fboxrule}{0.03cm}
	\fbox{\parbox[t][3.8cm][t]{\textwidth}{
		{\footnotesize \textbf{Eingangsstempel Infothek\\
				Studierendenverwaltung | Prüfungsamt}}}}
\end{minipage}

\vspace{1cm}