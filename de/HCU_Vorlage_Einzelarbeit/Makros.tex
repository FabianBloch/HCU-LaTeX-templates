% Vorlagen für einheitliche Formatierungen:

% Überlange Überschrift
\section[Überschrift]{\texorpdfstring{Überschrift tex}{Überschrift pdf}}


% Textfarbe
\textcolor{HCU}{}
\textcolor{red1}{}


% Anführungszeichen
\Quotationmarks{text}


% 2 Quellen
(\citeauthor{}, \citeyearNP{}, S. ; \citeauthor{}, \citeyearNP{}, S. )


% Abbildung mit Quelle
\begin{figure}[H]
	\centering
	\includegraphics[width=0.75\textwidth]{Daten/Datei}
	\caption{Überschrift}
	\caption*{Quelle: \citeA{}}
	\label{fig:}
\end{figure}


% Bild mit trim
\begin{figure}[H]
	\centering
	% trim = left lower right upper
	\includegraphics[trim={95mm 17mm 110mm 15mm}, clip, width=1\textwidth]{Daten/DGM.png}
	\caption{Überschrift}
	\caption*{Quelle: \citeA{}}
	\label{fig:}
\end{figure}


% 2 Abbildungen mit Quelle
\begin{figure}[H]
	\begin{subfigure}[c]{0.48\textwidth}
		\includegraphics[width=\textwidth]{Daten/}
		\subcaption{}
		\label{fig:}
	\end{subfigure}
	\hfill
	\begin{subfigure}[c]{0.48\textwidth}
		\includegraphics[width=\textwidth]{Daten/}
		\subcaption{}
		\label{fig:}
	\end{subfigure}
	\caption{}
	\caption*{Quelle: \citeA[]{}}
	\label{fig:}
\end{figure}


% Formel mit Nummerierung
\begin{equation}
	\numberwithin{equation}{section}
	Formel \label{eq:Label} \\
\end{equation}


% Python-Input aus Datei
\lstinputlisting[language=Python, style=Python, firstline=, lastline=, caption=Überschift, label={lst:Label}]{Daten/Datei.py}


% Aufzählung
\begin{itemize}	
	\setlength{\itemsep}{-2pt}
	\item 
	\item 
	\item 
\end{itemize}


% minipage
\begin{minipage}[H]{0.48\textwidth}
	
\end{minipage}
\hfill
\begin{minipage}[H]{0.48\textwidth}
	
\end{minipage}


% Spalten
\begin{multicols}{2}
	
\end{multicols}


% eine PDF-Seite hinzufügen
\begin{center} %trim= left bottom right top
	\includegraphics[trim = 0mm 0mm 0mm 0mm,clip, page = 1, width=1\textwidth]{Daten/Datei.pdf}\\
\end{center}


% PDF als Raster hinzufügen
\includepdf[pages={},nup= 2x5,delta=3mm 3mm,pagecommand={},width=0.5\textwidth]{Daten/Datei}